\documentclass[11pt]{article}
\usepackage[margin=1in]{geometry}
\usepackage{amsmath,amssymb}
\usepackage{hyperref}
\usepackage{mathtools}
\usepackage{physics}
\usepackage{siunitx}
\usepackage{enumitem}
\setlist{nosep}

\title{Physics-and-Programming: A Compact Equation Companion}
\author{}
\date{}

\begin{document}
\maketitle

\begin{abstract}
This document provides a research-note style companion to the visualizations in the \texttt{QuantumStuff} folder. Each section summarizes the physical idea, gives the governing equations used in the code, and adds brief interpretive remarks or examples. The goal is clarity rather than completeness; the formulas are faithful to the scripts and are intended for intuition-building and qualitative analysis.
\end{abstract}

\section*{Notation and Conventions}
We use $x,y$ for spatial coordinates, $t$ for time, $\vec r$ for position, and $\vec v$ for velocity. Dots indicate time derivatives, e.g. $\dot{x}=dx/dt$. Constants are chosen in the scripts for visual clarity; many figures use normalized units where $\hbar=1$, $c=1$, or $m=1$. Unless otherwise stated, equations are not claimed to be high-precision physical models.

\section*{QuantumStuff/4th\_dimension\_viz.py (Gravitational Lensing Toy Sheet)}
This visualization treats curved spacetime as a 2D surface with a central potential well and traces a bent light path. The warping function is normalized to keep the edges near zero while deepening near the center. The light ray is a base line plus a Gaussian bend term.
\[
 r=\sqrt{x^2+y^2}
\]
\[
 z_0=-\frac{W}{\sqrt{r^2+s^2}},\quad z=-W\cdot\frac{z_0-z_{0,\min}}{0-z_{0,\min}}
\]
\[
 x_{\text{base}}(t)=x_s+(x_e-x_s)t,\quad y_{\text{base}}(t)=y_s+(y_e-y_s)t
\]
\[
 y_{\text{path}}=y_{\text{base}}-1.3\,e^{-(x_{\text{base}})^2/(2\cdot1.1^2)}
\]
Example: increasing $W$ deepens the well and makes curvature more dramatic, even though the ray is still a scripted path.

\section*{QuantumStuff/atomic\_bomb\_viz.py (Stylized Chain Reaction)}
This is a non-physical, artistic particle animation. Motion is simple advection with wall bounces; "splitting" triggers are randomized collisions.
\[
 x_{t+1}=x_t+v_x,\quad y_{t+1}=y_t+v_y
\]
\[
 v_x\to -v_x \text{ if } x\notin[X_{\min},X_{\max}],\quad v_y\to -v_y \text{ if } y\notin[Y_{\min},Y_{\max}]
\]
\[
 r=U^{0.6}(0.7R),\quad dx=r\cos\theta,\quad dy=r\sin\theta
\]
\[
 \text{hit if } (x-x_n)^2+(y-y_n)^2\le (R+0.12)^2
\]

\section*{QuantumStuff/atom\_viz.py (Electron Cloud Toy)}
Electrons move on noisy circular paths to convey "cloud-like" behavior. The nucleus is a small random cluster.
\[
 r=R\,U^{1/3},\quad x=r\cos\theta,\quad y=r\sin\theta
\]
\[
 \theta_{t+1}=\theta_t+\omega\cdot0.03,\quad r(t)=r_0+A\sin\phi
\]
\[
 x=r\cos\theta,\quad y=r\sin\theta
\]
Example: increasing jitter amplitude $A$ makes the electron cloud appear more diffuse.

\section*{QuantumStuff/blackhole\_viz.py (Cinematic Black Hole)}
A synthetic field combines a lensed starfield, a photon ring, and accretion disk textures. The equations are stylized but capture qualitative features like beaming and falloff.
\[
 \alpha=\frac{L}{r^2},\quad (x',y')=(x,y)+\alpha\left(\frac{x}{r},\frac{y}{r}\right)
\]
\[
 \text{disk radial}=\exp\left(-\frac{(R-R_d)^2}{W^2}\right)
\]
\[
 \text{doppler}=0.32+0.68\cdot\frac{1+\cos(\theta-\Omega t)}{2}
\]
\[
 \text{spiral}=0.5+0.5\cos\left(m(\theta-1.4\Omega t)\right)
\]
\[
 \text{hole falloff}=1-\exp\left[-\left(\frac{R}{R_s}\right)^3\right]
\]

\section*{QuantumStuff/collision\_bh\_viz.py (Binary Merger Toy)}
Two softened potentials orbit inward and merge. A sinusoidal radial ripple imitates gravitational waves.
\[
 \Phi(x,y)=\frac{1}{\sqrt{(x-x_c)^2+(y-y_c)^2+\epsilon^2}}
\]
\[
 x_{1,2}=\mp \frac{s}{2}\cos\phi,\quad y_{1,2}=\mp \frac{s}{2}\sin\phi
\]
\[
 \text{ring}=\exp\left(-\frac{(r-r_0)^2}{w^2}\right)
\]
\[
 \text{GW ripple}=e^{-\gamma|R-r_0|}\sin(k(R-r_0))
\]

\section*{QuantumStuff/dark\_matter\_viz.py (Halo Intuition)}
A smooth halo profile extends beyond the luminous disk to suggest flat rotation curves. Tracers follow roughly constant angular speeds.
\[
 I(r)=\frac{1}{1+(r/r_c)^2}+\frac{1.8}{1+(r/R_h)^3}
\]
\[
 \theta_{t+1}=\theta_t+\omega,\quad x=r\cos\theta,\quad y=r\sin\theta
\]
\[
 (dx,dy)=\eta\frac{(x,y)}{r^2}
\]

\section*{QuantumStuff/doppler\_eff\_viz.py (Doppler Effect)}
Wavefronts are emitted periodically from a moving source. In front of the source, crests are compressed; behind, they spread out.
\[
 x_s(t)=x_0+v_s t
\]
\[
 r_{\text{out}}=c(t-t_0),\quad r_{\text{in}}=R_0-c(t-t_0)
\]

\section*{QuantumStuff/double\_pendulum\_chaos\_viz.py (Chaotic Dynamics)}
The standard double-pendulum equations are integrated with RK4. A tiny initial angle difference leads to exponential divergence.
\[
 x_1=L_1\sin\theta_1,\quad y_1=-L_1\cos\theta_1
\]
\[
 x_2=x_1+L_2\sin\theta_2,\quad y_2=y_1-L_2\cos\theta_2
\]
\[
 \ddot{\theta}_1=\frac{M_2L_1\omega_1^2\sin\Delta\cos\Delta+M_2g\sin\theta_2\cos\Delta+M_2L_2\omega_2^2\sin\Delta-(M_1+M_2)g\sin\theta_1}{(M_1+M_2)L_1-M_2L_1\cos^2\Delta}
\]
\[
 \ddot{\theta}_2=\frac{-M_2L_2\omega_2^2\sin\Delta\cos\Delta+(M_1+M_2)(g\sin\theta_1\cos\Delta-L_1\omega_1^2\sin\Delta-g\sin\theta_2)}{(L_2/L_1)((M_1+M_2)L_1-M_2L_1\cos^2\Delta)}
\]

\section*{QuantumStuff/double\_slit\_viz.py (Particle Double Slit)}
Particles are launched, checked against slit openings, and if transmitted are spread by a Gaussian "diffraction" jitter.
\[
 y_b=y_0+\mathcal{N}(0,\sigma_b)
\]
\[
 \text{pass if } |y_b-\pm y_s|\le \frac{h}{2}
\]
\[
 y_{\text{screen}}=y_b+\mathcal{N}(0,\sigma_d)
\]

\section*{QuantumStuff/electric charges interaction.py (Coulomb Field)}
The electric field is computed from two charges and visualized with streamlines and quivers. Force vectors are shown for intuition.
\[
 \vec E(\vec r)=kq\frac{\vec r-\vec r_q}{\left(|\vec r-\vec r_q|^2+\epsilon^2\right)^{3/2}}
\]
\[
 \vec F_{12}=k\frac{q_1q_2}{r^2}\hat r,\quad \vec F_{21}=-\vec F_{12}
\]

\section*{QuantumStuff/electric field.py (Single Charge Field)}
A softened point-charge field is used to avoid singularity at the origin.
\[
 \vec E(\vec r)=kq\frac{\vec r}{\left(r^2+\epsilon^2\right)^{3/2}}
\]

\section*{QuantumStuff/enthropy\_viz.py (Shannon Entropy)}
Particles diffuse randomly in a box while the Shannon entropy of the 2D histogram rises.
\[
 \vec x_{t+1}=\vec x_t+\mathcal{N}(0,\sigma^2)
\]
\[
 H=-\sum_i p_i\log_2 p_i
\]

\section*{QuantumStuff/fission\_fusion\_viz.py (Stylized Nuclear Processes)}
This is a safe, non-physical visualization contrasting fission-like and fusion-like motions.
\[
 x_{t+1}=x_t+v_x,\quad y_{t+1}=y_t+v_y,\quad \vec v\to 0.97\vec v
\]
\[
 \text{split if } (x-x_n)^2+(y-y_n)^2\le (R+\delta)^2
\]
\[
 \text{fusion if } (x_a-x_b)^2+(y_a-y_b)^2\le d_f^2 \text{ and } U<p_f
\]

\section*{QuantumStuff/fluid\_vortex\_viz.py (K\'arm\'an Vortex Street Toy)}
A synthetic velocity field mimics flow past a cylinder with alternating wake shedding.
\[
 \vec u=\vec u_{\text{base}}+\vec u_{\text{deflect}}+\vec u_{\text{wake}}
\]
\[
 \text{deflect}=\frac{R_c^2}{r^2},\quad u_d=-1.2(xy)\,\text{deflect}
\]
\[
 v_d=0.6\left(1-\frac{2y^2}{r^2}\right)\text{deflect}
\]
\[
 u_{\text{wake}}=0.25e^{-0.6(x-R_c)}\sin(\pi y),\quad v_{\text{wake}}=0.6e^{-0.6(x-R_c)}\sin\left(2\pi(ft-0.35x)\right)
\]

\section*{QuantumStuff/gravitational\_lensing\_viz.py (Point-Mass Lens)}
The standard lens equation is used to remap a background texture as the Einstein radius varies.
\[
 \boldsymbol{\beta}=\boldsymbol{\theta}-\theta_E^2\frac{\boldsymbol{\theta}}{|\boldsymbol{\theta}|^2}
\]

\section*{QuantumStuff/hawkin's\_rad\_viz.py (Hawking Radiation Toy)}
A bright halo and expanding rings suggest radiation escaping the horizon.
\[
 \text{glow}=\exp\left(-\frac{(R-R_h)^2}{w^2}\right)
\]
\[
 R_{\text{wave}}=R_h+v(t-t_0),\quad \text{ring}=\exp\left(-\frac{(R-R_{\text{wave}})^2}{\sigma^2}\right)
\]

\section*{QuantumStuff/hawking\_particles\_escape\_viz.py (Particle Escape Toy)}
Discrete particles arc outward with mild angular drift.
\[
 r=R_h+v(t-t_0)
\]
\[
 \theta=\theta_0+s\kappa\left(1-e^{-\beta(r-R_h)}\right)+0.08\sin(\omega t+\phi)
\]
\[
 x=r\cos\theta,\quad y=r\sin\theta
\]

\section*{QuantumStuff/hydrogen\_bomb\_viz.py (Stylized Fusion)}
A decorative "fusion" animation with proximity checks and luminous flashes.
\[
 x_{t+1}=x_t+v_x,\quad y_{t+1}=y_t+v_y
\]
\[
 \vec v\leftarrow \vec v+0.02\frac{-\vec r}{r} \text{ if } r>r_{\max}
\]
\[
 \text{fusion if } (x_i-x_j)^2+(y_i-y_j)^2\le d_f^2 \text{ and } U<p_f
\]

\section*{QuantumStuff/interferometer\_gw\_viz.py (GW Interferometer)}
A Michelson-like setup shows arm-length modulation and the resulting phase shift.
\[
 h(t)=A\sin(2\pi f t)
\]
\[
 L_x=L(1+\tfrac{1}{2}h),\quad L_y=L(1-\tfrac{1}{2}h)
\]
\[
 \Delta\phi=\frac{4\pi}{\lambda}(L_x-L_y)
\]
\[
 I=\frac{1}{2}\left(1+\cos\Delta\phi\right)
\]

\section*{QuantumStuff/maxwell\_wave\_viz.py (Plane EM Wave)}
A 1D plane wave satisfies Maxwell's equations in vacuum; $E$ and $B$ are in phase and orthogonal.
\[
 E_y(x,t)=E_0\sin(kx-\omega t)
\]
\[
 B_z(x,t)=\frac{E_0}{c}\sin(kx-\omega t),\quad \omega=ck
\]

\section*{QuantumStuff/newton\_laws\_viz.py (Newton's Laws)}
The animation shows inertia, $F=ma$, and equal-and-opposite impulses in three panels.
\[
 x(t)=x_0+vt\quad (\Sigma F=0)
\]
\[
 a=\frac{F}{m},\quad v_{t+1}=v_t+a\Delta t,\quad x_{t+1}=x_t+v_{t+1}\Delta t
\]
\[
 \vec J_1=-\vec J_2
\]

\section*{QuantumStuff/particle\_acc\_viz.py (Circular Accelerator Toy)}
Particles move on a ring; collision events are triggered near a fixed interaction point.
\[
 \theta_{t+1}=\theta_t+\omega,\quad x=x_0+r\cos\theta,\quad y=y_0+r\sin\theta
\]
\[
 \text{collision if } |\Delta\theta|<\theta_{\text{window}}
\]

\section*{QuantumStuff/particle\_entang\_viz.py (Entanglement Toy)}
Pairs fly apart and "collapse" to correlated outcomes when either hits a detector line.
\[
 x_{t+1}=x_t+v_x,\quad y_{t+1}=y_t+v_y
\]
\[
 \text{collapse when } x\le x_L \text{ or } x\ge x_R,\quad s_L=-s_R
\]

\section*{QuantumStuff/Quantum.py (Classical Bit vs Qubit)}
The qubit state is parameterized on a Bloch circle cross-section.
\[
 |\psi\rangle=\cos\frac{\theta}{2}|0\rangle+e^{i\phi}\sin\frac{\theta}{2}|1\rangle
\]
\[
 P(0)=|a|^2,\quad P(1)=|b|^2
\]
\[
 n_x=\sin\theta,\quad n_z=\cos\theta
\]

\section*{QuantumStuff/quantum\_particle\_viz.py (2D Superposition)}
A sum of stationary states produces a time-varying probability density.
\[
 \psi(x,y,t)=\sin(\pi x)\sin(\pi y)e^{-i2\pi t}+0.6\sin(2\pi x)\sin(\pi y)e^{i1.5\pi t}+0.4\sin(\pi x)\sin(2\pi y)e^{-i\pi t}
\]
\[
 \rho=|\psi|^2,\quad \rho\leftarrow \rho/\sum\rho
\]

\section*{QuantumStuff/quantum\_search.py (Grover Toy)}
The success probability oscillates with iteration count.
\[
 \sin\theta=\frac{1}{\sqrt{N}}
\]
\[
 P_{\text{classical}}(k)=\min\left(\frac{k}{N},1\right)
\]
\[
 P_{\text{Grover}}(k)=\sin^2\left((2k+1)\theta\right)
\]
\[
 k^*\approx \frac{\pi}{4\theta}-\frac{1}{2}
\]

\section*{QuantumStuff/quantum\_slit\_wave\_viz.py (Wave Double Slit)}
Two slit sources interfere to form a spatial intensity pattern.
\[
 \psi_{\text{in}}=\cos\left(k(x-x_0)-\omega t\right)
\]
\[
 \psi_{1,2}=\frac{\cos(kr_{1,2}-\omega t)}{r_{1,2}^p}
\]
\[
 I=(\psi_1+\psi_2)^2
\]

\section*{QuantumStuff/quantum\_tunneling\_viz.py (Split-Step Method)}
A Gaussian packet interacts with a rectangular barrier; evolution uses split-step Fourier propagation.
\[
 \psi(x,0)=\left(\frac{1}{\pi\sigma^2}\right)^{1/4}e^{-(x-x_0)^2/(2\sigma^2)}e^{ik_0x}
\]
\[
 V(x)=\begin{cases}
V_0,& |x|<\tfrac{w}{2}\\
0,& \text{otherwise}
\end{cases}
\]
\[
 \psi(t+\Delta t)=e^{-iV\Delta t/2}\,\mathcal{F}^{-1}\left[e^{-ik^2\Delta t/2}\,\mathcal{F}\left(e^{-iV\Delta t/2}\psi(t)\right)\right]
\]
\[
 P_{\text{left}}=\int_{x<-5}|\psi|^2\,dx,\quad P_{\text{right}}=\int_{x>5}|\psi|^2\,dx
\]

\section*{QuantumStuff/quantum\_wave\_viz.py (Dispersing Packet)}
A Gaussian packet drifts and spreads, illustrating dispersion.
\[
 \sigma(t)=\sqrt{\sigma_0^2+(Dt)^2}
\]
\[
 x_c=x_0+Vt
\]
\[
 \psi(x,t)=\exp\left[-\frac{(x-x_c)^2}{2\sigma(t)^2}\right]e^{i(K_0(x-x_c)-\omega t)}
\]

\section*{QuantumStuff/relativistic\_time\_dilation\_viz.py (Proper Time)}
Proper time accumulates more slowly for a moving clock.
\[
 x(t)=x_0+A\sin(\omega t),\quad v=A\omega\cos(\omega t)
\]
\[
 \beta=\frac{v}{c},\quad \gamma=\frac{1}{\sqrt{1-\beta^2}}
\]
\[
 d\tau=\frac{dt}{\gamma}
\]

\section*{QuantumStuff/schrodinger\_cat\_viz.py (Decoherence Toy)}
Coherence decays exponentially until collapse.
\[
 \mathcal{C}(t)=e^{-\Gamma t}
\]
\[
 P_{\text{alive}}=0.5+0.15\sin(0.8t),\quad P_{\text{dead}}=1-P_{\text{alive}}
\]

\section*{QuantumStuff/thermodynamics\_laws\_viz.py (Thermodynamics)}
The panels illustrate equilibrium, energy bookkeeping, entropy increase, and the third law limit.
\[
 T_i(t)=T_{\text{target}}+\Delta_i e^{-0.35t}
\]
\[
 \Delta U=Q-W
\]
\[
 f_{\text{mix}}=1-e^{-0.25t}
\]
\[
 T(t)=320e^{-0.012t}+2,\quad S=\ln(T+1)
\]

\section*{QuantumStuff/uncertainty\_wavepacket\_viz.py (Uncertainty Principle)}
A breathing Gaussian illustrates $\Delta x\,\Delta p\gtrsim \hbar/2$.
\[
 |\psi(x)|=e^{-\frac{(x-\mu)^2}{2\sigma_x^2}}
\]
\[
 |\phi(p)|=e^{-\frac{(p-p_0)^2}{2\sigma_p^2}},\quad \sigma_p=\frac{1}{2\sigma_x}\; (\hbar=1)
\]
\[
 \Delta x\,\Delta p\approx \sigma_x\sigma_p
\]

\section*{QuantumStuff/wormhole\_viz.py (Wormhole Toy)}
Particles spiral into a central throat and re-emerge on the other side.
\[
 \theta_{t+1}=\theta_t+\omega,\quad r_{t+1}=r_t-v_r
\]
\[
 x=x_c+r\cos\theta,\quad y=y_c+r\sin\theta
\]
\[
 \text{warp factor}=\frac{1}{1+0.4/r^2}
\]

\section*{QuantumStuff/simple\_harmonic\_oscillator\_viz.py (Mass--Spring)}
A classic oscillator with total energy conserved.
\[
\ddot{x} + \frac{k}{m} x = 0,\quad \omega_0 = \sqrt{\frac{k}{m}}
\]
\[
E = \tfrac{1}{2} m v^2 + \tfrac{1}{2} k x^2
\]

\section*{QuantumStuff/damped\_forced\_oscillator\_viz.py (Driven Oscillator)}
Damping removes energy while the drive injects it, yielding steady-state oscillations.
\[
\ddot{x} + 2\beta \dot{x} + \omega_0^2 x = \frac{F_0}{m} \cos(\omega_d t)
\]

\section*{QuantumStuff/angular\_momentum\_viz.py (Central Force)}
In a central potential, angular momentum is conserved.
\[
\vec F = -k \vec r,\quad \vec L = m(\vec r \times \vec v)
\]
\[
L_z = m(x v_y - y v_x),\quad \frac{d\vec L}{dt} = \vec r \times \vec F = 0
\]

\section*{QuantumStuff/projectile\_drag\_viz.py (Quadratic Drag)}
A projectile under gravity and drag follows a shorter, asymmetric trajectory.
\[
 m \frac{d\vec v}{dt} = -m g \, \hat{y} - c\,|\vec v|\,\vec v
\]

\section*{QuantumStuff/two\_body\_orbit\_viz.py (Newtonian Two-Body)}
Two masses interact via inverse-square gravity, tracing bound orbits around the barycenter.
\[
 m_1 \ddot{\vec r}_1 = G m_1 m_2 \frac{\vec r_2 - \vec r_1}{|\vec r_2 - \vec r_1|^3}
\]
\[
 m_2 \ddot{\vec r}_2 = G m_1 m_2 \frac{\vec r_1 - \vec r_2}{|\vec r_1 - \vec r_2|^3}
\]

\section*{QuantumStuff/fluid\_mechanics\_channel\_viz.py (Poiseuille Flow)}
Laminar flow in a channel yields a parabolic velocity profile.
\[
 u(y) = U_{\max}\left(1 - \left(\frac{y}{H}\right)^2\right),\quad v=0
\]

\section*{QuantumStuff/maxwell\_equations\_viz.py (Maxwell's Equations)}
A plane electromagnetic wave in vacuum satisfies the full Maxwell system.
\[
\nabla\cdot\vec E = 0
\]
\[
\nabla\cdot\vec B = 0
\]
\[
\nabla\times\vec E = -\frac{\partial \vec B}{\partial t}
\]
\[
\nabla\times\vec B = \mu_0\varepsilon_0\frac{\partial \vec E}{\partial t}
\]

\end{document}

